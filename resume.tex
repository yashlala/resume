\documentclass{resume} % Use the custom resume.cls style

\usepackage[left=0.35in,top=0.35in,right=0.35in,bottom=0.35in]{geometry} % Document margins
\newcommand{\tab}[1]{\hspace{.2667\textwidth}\rlap{#1}}
\newcommand{\itab}[1]{\hspace{0em}\rlap{#1}}
\usepackage{titlesec}
\titlespacing\section{0pt}{6pt plus 4pt minus 2pt}{0pt plus 2pt minus 2pt}
\titlespacing\subsection{0pt}{6pt plus 4pt minus 2pt}{0pt plus 2pt minus 2pt}
\titlespacing\subsubsection{0pt}{6pt plus 4pt minus 2pt}{0pt plus 2pt minus 2pt}

\name{Yash Lala}
\address{Palo Alto, CA}
\address{(510)-400-5572 \\ yashlala@ucla.edu \\ https://yashlala.com/}

\showboxdepth=\maxdimen
\showboxbreadth=\maxdimen

\begin{document}

\begin{resumeSection}{Education}

UCLA B.S. in Computer Science
	\hfill {GPA: 3.782, 2018 - 2022} \\
BASIS Independent Silicon Valley High School
	\hfill {GPA: 3.9, 2014 - 2018}

\end{resumeSection}


\begin{resumeSection}{Technical Skills}

\begin{tabular}{ @{} >{\bfseries}l @{\hspace{6ex}} l }
Computer Languages &  C, Python, Shell, Java, Go, C++, OCaml, SQL. \\
Software \& Tools & Ansible, Git, GCP, etc. Strong focus on kernel development
and debugging. 
\end{tabular}

\end{resumeSection}


\begin{resumeSection}{Professional Experience} \itemsep -10pt

\begin{resumeSubsection}{Pringle Lab, Stanford Genetics Department}
	{June 2017 - August 2017}{Undergraduate Research Intern}{}
\item Tested for selective lectin protein binding to various algal species,
	in order to understand the chemical processes behind coral bleaching. 
	Developed a microscopic cell image recognition+counting program for
	use in algal haemocytometry.
\end{resumeSubsection}
\begin{resumeSubsection}{Veritas Technologies LLC}
	{June 2021 - Sept 2021}{SDE Intern}{}
\item Worked on NetBackup Flex platform. Team tasked with implementing
	automatic compute node discovery and assimilation over a network.
	Replaced session-based internode communication scheme to an
	Ansible+HTTP based setup; replaced product-specific software components
	with platform-agnostic versions. 
\end{resumeSubsection}
\begin{resumeSubsection}{SOLAR Lab, UCLA CS Department} 
	{June 2022 - Present}{Student Researcher}{}
\item Independently developed patches for the Linux kernel's swap subsystems,
	with the goal of merging these changes upstream. The patchset allows
	for per-cgroup control of active swap devices. The changes have
	positive implications for swap throughput -- but more importantly, they
	make it easy to develop and control frontswap-based remote memory
	systems.
\item Prior to full-time employment: profiled swapout latencies for RDMA-based
	remote memory systems under various workloads and prefetch strategies.
	Worked on improving the Linux kernel's physical memory page allocation
	latency via asynchronous RCU-based refill of the per-cpu page lists. 
\end{resumeSubsection}

\end{resumeSection}

\begin{resumeSection}{Projects} \itemsep -10pt

\begin{resumeSubsection}{bNEAT}{September 2017 - May 2018}{}{}
\item Worked on developing an improved version of the Neuroevolution of
	Augmenting Topologies algorithm by recognizing and cloning distinct
	neural "subnets". Resulting algorithm runs through the initial learning
	phase faster than 'vanilla' NEAT.
\end{resumeSubsection}
\begin{resumeSubsection}{Junknet: Distributed Compilation Framework}
	{January 2021 - March 2021}{}{}
\item Worked on developing a distributed computing framework for a home
	environment. Project analyzes Makefiles and runs them in a distributed
	manner over available LAN devices. Network and device failures are
	tolerated.
\end{resumeSubsection}
\begin{resumeSubsection}{GRU4RecBE: Session Based Recommendations with Features}
	{March 2021 - June 2021}{}{}
\item Developed session-based recommendation system which extends the GRU4REC
	architecture with rich item features extracted from the pre-trained
	BERT architecture. Non-attentive model outperforms state-of-the-art
	session-based models over the benchmark MovieLens 1M and MovieLens 20M
	datasets. Paper accepted to AAAI Student track. 
\end{resumeSubsection}

\end{resumeSection}

\end{document}
