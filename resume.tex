\documentclass{resume} % Use the custom resume.cls style

\usepackage[left=0.35in,top=0.35in,right=0.35in,bottom=0.35in]{geometry} % Document margins
\newcommand{\tab}[1]{\hspace{.2667\textwidth}\rlap{#1}}
\newcommand{\itab}[1]{\hspace{0em}\rlap{#1}}
\usepackage{titlesec}
\titlespacing\section{0pt}{6pt plus 4pt minus 2pt}{0pt plus 2pt minus 2pt}
\titlespacing\subsection{0pt}{6pt plus 4pt minus 2pt}{0pt plus 2pt minus 2pt}
\titlespacing\subsubsection{0pt}{6pt plus 4pt minus 2pt}{0pt plus 2pt minus 2pt}

\name{Yash Lala}
\address{Palo Alto, CA}
\address{(510)-400-5572 \\ yashlala@ucla.edu \\ https://yashlala.github.io/}

\showboxdepth=\maxdimen
\showboxbreadth=\maxdimen

\begin{document}

%\begin{resumeSection}{Overview and Availability}
%
%I am a fourth-year computer science undergraduate student at UCLA. My interests
%include parallel computation and distributed system design + debugging. I am
%looking for a part-time research assistency (availability 14-19 hours per week)
%over the course of the academic year. 
%
%\end{resumeSection}

\begin{resumeSection}{Education}

UCLA B.S. in Computer Science (ongoing)
	\hfill {GPA: 3.754, 2018 - Present} \\
BASIS Independent Silicon Valley High School
	\hfill {GPA: 3.9, 2014 - 2018}

\end{resumeSection}


\begin{resumeSection}{Technical Skills}

\begin{tabular}{ @{} >{\bfseries}l @{\hspace{6ex}} l }
Computer Languages &  C, Python, Shell, Java, Go, C++, OCaml, SQL. \\
Software \& Tools & Ansible, Git, GCP, etc. Strong focus on scripting and kernel mechanisms. \\
\end{tabular}

\end{resumeSection}


\begin{resumeSection}{Relevant Coursework}

\itab{CS 111: Operating Systems} \tab{}
	\tab{CS 118: Computer Networks} \\
\itab{CS 131: Programming Languages Architecture} \tab{}
	\tab{CS 134: Distributed Systems} \\
\itab{CS 145 \& 247: Intro \& Advanced Data Mining} \tab{}
	\tab{CS 180: Algorithms} \\
 \itab{CS 181: Formal Languages and Automata Theory} \tab{}
	\tab{115AH: Linear Algebra} \\
\itab{CS 130: Software Engineering} \tab{}
	\tab{CS 143: Database Systems} \\
\itab{CS 132: Compiler Construction} \tab{}
	\tab{CS 214: Big Data Systems}

\end{resumeSection}


\begin{resumeSection}{Relevant Experience} \itemsep -10pt

\begin{resumeSubsection}{Pringle Lab, Stanford Genetics Department}
	{June 2017 - August 2017}{Undergraduate Research Intern}{}
\item Tested for selective lectin binding to various algal species. Developed a
	microscopic cell image recognition+counting program for use in algal
	haemocytometry.
\end{resumeSubsection}
\begin{resumeSubsection}{Veritas Technologies LLC}
	{June 2021 - Sept 2021}{SDE Intern}{}
\item Worked on NetBackup Flex platform. Team tasked with implementing
	automatic compute node discovery and assimilation over a network.
	Refactored internal logic: replaced session-based internode
	communication scheme to an Ansible+HTTP based setup; replaced
	product-specific software components with platform-agnostic versions. 
\end{resumeSubsection}

\end{resumeSection}


\begin{resumeSection}{Independent Projects} \itemsep -10pt

\begin{resumeSubsection}{bNEAT}{September 2017 - May 2018}{}{}
\item Worked on developing an improved version of the Neuroevolution of
	Augmenting Topologies algorithm by recognizing and cloning distinct
	neural "subnets". Resulting algorithm runs through the initial learning
	phase faster than 'vanilla' NEAT.
\end{resumeSubsection}
\begin{resumeSubsection}{Junknet: Distributed Compilation Framework}
	{January 2021 - March 2021}{}{}
\item Worked on developing a distributed computing framework for a home
	environment. Project analyzes Makefiles and runs them in a distributed
	manner over available LAN devices. Network and device failures are
	tolerated.
\end{resumeSubsection}
\begin{resumeSubsection}{GRU4REC-F: Session Based Recommendations with Features}
	{March 2021 - June 2021}{}{}
\item Developed session-based recommendation system which extends the GRU4REC
	architecture with rich item features extracted from the pre-trained
	BERT architecture. Non-attentive model outperforms state-of-the-art
	session-based models over the benchmark MovieLens 1M and MovieLens 20M
	datasets. Paper accepted to AAAI Student track. 
\end{resumeSubsection}

\end{resumeSection}

\end{document}
