\documentclass{resume} % Use the custom resume.cls style

\usepackage[left=0.35in,top=0.35in,right=0.35in,bottom=0.35in]{geometry} % Document margins
\newcommand{\tab}[1]{\hspace{.2667\textwidth}\rlap{#1}}
\newcommand{\itab}[1]{\hspace{0em}\rlap{#1}}
\usepackage{titlesec}

\usepackage{amsmath}

\usepackage{hyperref}
\hypersetup{
	colorlinks=true,
	linkcolor=blue, 
	urlcolor=cyan,
}
\urlstyle{same}

\titlespacing\section{0pt}{6pt plus 4pt minus 2pt}{0pt plus 2pt minus 2pt}
\titlespacing\subsection{0pt}{6pt plus 4pt minus 2pt}{0pt plus 2pt minus 2pt}
\titlespacing\subsubsection{0pt}{6pt plus 4pt minus 2pt}{0pt plus 2pt minus 2pt}

\name{Yash Lala}
\address{(510)-400-5572 \\ 
	\href{mailto://yashlala@ucla.edu/}{yashlala@ucla.edu} \\
	Palo Alto, CA}
\address{\href{https://yashlala.com/}{yashlala.com} \\ 
	\href{https://github.com/yashlala}{github.com/yashlala} \\ 
	\href{https://linkedin.com/in/yashlala}{linkedin.com/in/yashlala}}

\showboxdepth=\maxdimen
\showboxbreadth=\maxdimen

\begin{document}

\begin{resumeSection}{Education}

UCLA B.S. in Computer Science
	\hfill GPA: 3.782, 2018 - 2022 \\
BASIS Independent Silicon Valley High School
	\hfill GPA: 3.9, 2014 - 2018

\end{resumeSection}

\begin{resumeSection}{Technical Skills}

\begin{tabular}{ @{} >{\bfseries}l @{\hspace{6ex}} l }
Programming Languages &  C, Python, Unix shells, Java, Go, C++, OCaml, SQL \\
Software \& Tools & QEMU + GDB, Linux kernel debugging, Docker, Ansible, PyTorch, Git, AWS, LaTeX
\end{tabular}

\end{resumeSection}


\begin{resumeSection}{Professional Experience} \itemsep -12pt

\begin{resumeSubsection}{SOLAR Lab, UCLA CS Department} 
	{Sept 2021 - Present}{Student Researcher}{Supervisor: Prof. Harry Xu}
\item Volunteered during the school year, employed full-time to work on kernel
	patches over the summer. Focused on developing OS kernel mechanisms to
	allow for transparent memory disaggregation. Worked heavily with kernel
	programming and debugging tools, such as \textbf{QEMU} + \textbf{GDB},
	serial port debugging, and \textbf{perf}. 
\item Independently developed patches for the Linux kernel's swap subsystems,
	with the goal of merging these changes upstream. Patchset extends the
	cpuset controller to allow per-cgroup control of active swap devices.
	Associated refactoring has positive implications for swap throughput,
	and makes it easy to manage frontswap-based remote memory systems. \\
	Code at 
	\href{https://github.com/yashlala/canvas-linux}{github.com/yashlala/canvas-linux}.
\item Developed a patchset to improve the Linux kernel's physical page
	allocation latency. The patch reduces tail latencies by refilling the
	percpu low-order free page lists asynchronously using RCU. 
\item Profiled swapout latencies for RDMA-based remote memory systems under
	various workloads and prefetch strategies. 
\end{resumeSubsection}

\begin{resumeSubsection}{CSSI Program, UCLA CS Department}
	{July 2022}{Tutor Undergrad (TA)}{}
\item Taught introductory data science to high school students for an intensive
	summer program. Led 4 hours of discussion section and office hours per
	day, prepared discussion material and assignments, graded papers, and
	advised students about careers in the field. 
\end{resumeSubsection}

\begin{resumeSubsection}{Veritas Technologies LLC}
	{June 2021 - Sept 2021}{SDE Intern}{}
\item Worked on large-scale data consolidation and backup devices (NetBackup
	Flex platform). 
\item Implemented functionality allowing Flex nodes to automatically discover
	new backup nodes over the datacenter network, then to securely
	assimilate them into a backup cluster. Primarily worked with
	\textbf{Ansible}, \textbf{Docker}, and various glue languages. 
\item Replaced SSH-based inter-node communication protocols with a RESTy HTTP
	based protocol. 
\item Added web dashboard for backup cluster management. 
\end{resumeSubsection}

\begin{resumeSubsection}{Pringle Lab, Stanford Genetics Department}
	{June 2017 - August 2017}{Undergraduate Research Intern}{}
\item Tested algal species for selective binding to various lectin proteins in
	order to understand the chemical processes behind coral bleaching.
	Poster available at
	\href{https://yashlala.com/pringle-poster.pdf}{yashlala.com/pringle-poster.pdf}. 
\item Developed an image recognition program in \textbf{Java} for use in
	algal cell haemocytometry.
\end{resumeSubsection}

\end{resumeSection}


\begin{resumeSection}{Projects} \itemsep -10pt

\begin{resumeSubsection}
	{$\text{SC-DNN}_{cc}$: A Compiler for Stochastic-Computing Accelerators}
	{May 2022 - June 2022}{}{}
\item Developed a compiler backend that transforms programs written in
	conventional IRs into forms that can be run on a stochastic-computing
	based hardware accelerator (stochastic accelerators have unusual
	probability-based programming semantics, and can be difficult to
	program). Developed an interpreter in \textbf{Java} to emulate a
	stochastic accelerator with a limited set of stochastic primitive
	operations. 
\end{resumeSubsection}

\begin{resumeSubsection}{NDN Multicast}{May 2022-Present}{}{}
\item Worked on extending routing protocols for NDN (Named Data Networks). 
	Extended NLSR (a link-state routing algorithm for NDN) to allow for
	efficient multicast delivery of NDN Interest packets. Worked primarily
	in \textbf{C++}, Student paper submitted to ICN 2022. 
\end{resumeSubsection}

\begin{resumeSubsection}{GRU4RecBE: Session Based Recommendations with Features}
	{March 2021 - June 2021}{}{}
\item Developed session-based recommendation system in \textbf{PyTorch} which
	extends the GRU4REC architecture with rich item features extracted from
	the pre-trained BERT architecture. Non-attentive model outperforms
	state-of-the-art session-based models over the benchmark MovieLens 1M
	and MovieLens 20M datasets. Paper accepted to AAAI Student track. 
\end{resumeSubsection}

\end{resumeSection}

\end{document}
