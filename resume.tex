\documentclass{resume} % Use the custom resume.cls style

\usepackage[left=0.35in,top=0.35in,right=0.35in,bottom=0.35in]{geometry} % Document margins
\newcommand{\tab}[1]{\hspace{.2667\textwidth}\rlap{#1}}
\newcommand{\itab}[1]{\hspace{0em}\rlap{#1}}
\name{Yash Lala}
\address{1293 Lennon Way \\ San Jose, CA 95125}
\address{(510)-400-5572 \\ yashlala@g.ucla.edu \\ https://github.com/yashlala}

\begin{document}


\begin{resumeSection}{Education}

UCLA Undergraduate in Computer Science \hfill {GPA: 3.83, 2018 - Present}
\\ BASIS Independent Silicon Valley High School \hfill {GPA: 3.9, 2014 - 2018}
% Minor in Linguistics \smallskip \\
\end{resumeSection}


\begin{resumeSection}{Technical Strengths}

\begin{tabular}{ @{} >{\bfseries}l @{\hspace{6ex}} l }
Operating Systems & Strong knowledge of mechanisms behind OS mechanisms \\
Software \& Tools & GCP, etc. Strong focus on Linux automation/scripting. \\
Other & 1000+ hours of independent CS/Math/Science reading through online research. 
\end{tabular}

\end{resumeSection}


\begin{resumeSection}{Work Experience}

\begin{resumeSubsection}{Pringle Lab, Stanford Genetics Department}
	{June 2017 - August 2017}{Undergraduate Research Intern}{}
\item Focused on automating lab tasks using software. Worked on developing a
	microscopic cell recognition+counting program from scratch for use in
	algal haemocytometry. Program successfully ran through input data
	efficiently. Final product not ported to existing lab microscopes due
	to time constraints. 
	algal haemocytometry. 
\end{resumeSubsection}
\begin{resumeSubsection}{Sensagrate Dev Labs}{June 2019 - September 2019}{}{}
\item Trained in image-recognition, with a particular focus on using OpenCV for
	traffic pattern recognition. 
\end{resumeSubsection}

\end{resumeSection}


\begin{resumeSection}{Independent Projects} \itemsep -2pt

\begin{resumeSubsection}{bNEAT}{September 2017 - May 2018}{}{}
\item Worked on developing an improved version of the Neuroevolution of
	Augmenting Topologies algorithm by using subnet recognition to
	implement software analogues to homeobox genes. Tested the modified
	algorithm's performance by teaching it to play Super Mario World®.
	Resulting algorithm runs through the initial learning phase faster than
	'vanilla' NEAT.
\end{resumeSubsection}
\begin{resumeSubsection}{Text-recognizing Refreshable Braille Display}
	{December 2016 - May 2017}{}{}
\item Worked on developing an E-Reader for the blind. Used GNU Ocrad to
	recognize printed text and translate it into Braille dots on a
	deformable electroactive polymer based 'display'. 
\end{resumeSubsection}

\end{resumeSection}


\begin{resumeSection}{Relevant Coursework}

\itab{\textbf{CS Courses}} \tab{}  
	\tab{\textbf{Selected Other Courses}}
\\ \itab{Distributed Systems, Computer Networking} \tab{} 
	\tab{Statistics}
\\ \itab{Data Structures, Programming Languages} \tab{}  
	\tab{Linear Algebra, Differential Equations} 
\\ \itab{Computer Architecture, Operating Systems} \tab{}  
	\tab{Software Design Labs} 
\\ \itab{Algorithms} \tab{} 
	\tab{Physics}

\end{resumeSection}

\end{document}
