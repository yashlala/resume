\documentclass{resume} % Use the custom resume.cls style

\usepackage[left=0.35in,top=0.35in,right=0.35in,bottom=0.35in]{geometry} % Document margins
\newcommand{\tab}[1]{\hspace{.2667\textwidth}\rlap{#1}}
\newcommand{\itab}[1]{\hspace{0em}\rlap{#1}}

\name{Yash Lala}
\address{1293 Lennon Way \\ San Jose, CA 95125}
\address{(510)-400-5572 \\ yashlala@ucla.edu \\ https://yashlala.github.io/}

\begin{document}

\begin{resumeSection}{Overview and Availability}

I am a third-year computer science undergraduate student at UCLA. My interests
include parallel computation, distributed system design, network debugging, and
protein folding. I am seeking a summer internship. 

\end{resumeSection}


\begin{resumeSection}{Education}

UCLA B.S. in Computer Science (ongoing) 
	\hfill {GPA: 3.7, 2018 - Present} \\
BASIS Independent Silicon Valley High School 
	\hfill {GPA: 3.9, 2014 - 2018}

\end{resumeSection}


\begin{resumeSection}{Technical Skills}

\begin{tabular}{ @{} >{\bfseries}l @{\hspace{6ex}} l }
Computer Languages &  C, C++, Go, OCaml, Python, Rust, Scheme, Shell. \\
Software \& Tools & LaTeX, Git, GCP, etc. Strong focus on scripting and kernel mechanisms. \\
\end{tabular}

\end{resumeSection}


\begin{resumeSection}{Relevant Coursework}

\itab{CS 111: Operating Systems} \tab{} 
	\tab{CS 118: Computer Networks} \\
\itab{CS 131: Programming Languages Architecture} \tab{} 
	\tab{CS 134: Distributed Systems} \\
\itab{CS 145: Introduction to Data Mining} \tab{} 
	\tab{CS 180: Algorithms} \\
\itab{CS 181: Formal Languages and Automata Theory} \tab{}
	\tab{33A + 115AH: Linear Algebra} \\
\itab{CS 130: Software Engineering} \tab{} 
	\tab{CS 132: Compiler Construction}


\end{resumeSection}


\begin{resumeSection}{Relevant Experience}

\begin{resumeSubsection}{Pringle Lab, Stanford Genetics Department}
	{June 2017 - August 2017}{Undergraduate Research Intern}{}
\item Focused on automating miscellaneous lab tasks using software. Developed a 
	microscopic cell image recognition+counting program from scratch for
	use in algal haemocytometry. 
\end{resumeSubsection}
\begin{resumeSubsection}{Sensagrate Dev Labs}{June 2019 - September 2019}{}{}
\item Trained in image-recognition, with a particular focus on using OpenCV for
	traffic pattern recognition. 
\end{resumeSubsection}
\begin{resumeSubsection}{RNA Lab}{September 2020 - Present}{}{}
\item Worked on developing low-latency methods of IP
Packet payload classification. 
\end{resumeSubsection}

\end{resumeSection}


\begin{resumeSection}{Independent Projects} \itemsep -2pt

\begin{resumeSubsection}{bNEAT}{September 2017 - May 2018}{}{}
\item Worked on developing an improved version of the Neuroevolution of
	Augmenting Topologies algorithm by using subnet recognition to
	implement software analogues to homeobox genes. Tested the modified
	algorithm's performance by teaching it to play Super Mario
	World\textsuperscript{®}. Resulting algorithm runs through the initial
	learning phase faster than 'vanilla' NEAT.
\end{resumeSubsection}
\begin{resumeSubsection}{Text-recognizing Refreshable Braille Display}
	{December 2016 - May 2017}{}{}
\item Worked on developing an E-Reader for the blind. Used GNU Ocrad to
	recognize printed text and translate it into Braille dots on a
	novel deformable electroactive polymer based 'display'. 
\end{resumeSubsection}

\end{resumeSection}

\end{document}
